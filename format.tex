% This file was modified by Claude Sammut for ICML-2002 from the file
% made available by Pat Langley, Claude-Nicolas Fiechter, Mehmet Goker,
% and Cynthia Thompson for ICML-2K and by Andrea Danyluk for ICML-2002

\documentstyle[icml2002,epsf]{article}
% If you rely on Latex2e packages, replace the above line with: 
% \documentclass{article}
% \usepackage{icml2002}
% \usepackage{epsf}

\begin{document} 

\twocolumn[
\icmltitle{Formatting and Submission Instructions for the Nineteenth \\ 
           International Conference on Machine Learning}

\icmlauthor{Pat Langley}{langley@isle.org}
\icmladdress{Institute for the Study of Learning and Expertise, 
             2164 Staunton Court, Palo Alto, CA 94306 USA}
\icmlauthor{Claude-Nicolas Fiechter}{fiechter@rtna.daimlerchrysler.com}
\icmlauthor{Mehmet G\"{o}ker}{goker@rtna.daimlerchrysler.com}
\icmladdress{DaimlerChrysler Research and Technology Center, 
             1510 Page Mill Road, Palo Alto, CA 94304 USA}
\icmlauthor{Cynthia Thompson}{cthomp@csli.stanford.edu}
\icmladdress{Computational Learning Laboratory, 
             Center for the Study of Language and Information, 
             Stanford University, Stanford, CA 94305 USA}
\icmlauthor{Andrea Danyluk}{andrea@cs.williams.edu}
\icmladdress{Department of Computer Science, Williams College,
             Williamstown, MA 01267 USA}
\icmlauthor{Claude Sammut}{claude@cse.unsw.edu.au}
\icmladdress{School of Computer Science and Engineering,
             University of New South Wales,
             Sydney, Australia}
\vskip 0.3in
]
 
\begin{abstract} 
The paper abstract should begin in the left column, 0.4~inches below the
final address. The heading `Abstract' should be centered, bold, and in
11~point type. The abstract body should use 10~point type, with a 
vertical spacing of 11~points, and should be indented 0.25~inches more
than normal on left-hand and right-hand margins. Insert 0.4~inches 
of blank space after the body. Keep your abstract brief, limiting it
to one paragraph and no more than six or seven sentences.
\end{abstract} 

\section{Format of the Paper} 
 
All submissions should follow the same format to ensure the printer
can reproduce them without problems and to let readers more easily
find the information that they desire. 

\subsection{Length and Dimensions}

Papers must not exceed eight (8) pages, including all figures, tables, 
references, and appendices. We will return to the authors any submissions
that exceed this page limit or that diverge significantly from the format 
specified herein.

The text of the paper should be formatted in two columns, with an
overall width of 6.75 inches, length of 9.0 inches, and 0.25 inches
between the columns. The left margin should be 0.75 inches and the top
margin 1.0 inch (2.54~cm). The right and bottom margins will depend on
whether you print on US letter or A4 paper. 

The paper body should be set in 10~point type with a vertical spacing of 
11~points. Please use Times Roman typeface throughout the text. 

\subsection{Title and Author Information}

The paper title should be set in 14~point bold type and centered between 
two horizontal rules that are 1~point thick, with 1.0~inch between the
top rule and the top edge of the page. Capitalize the first letter of 
content words and put the rest of the title in lower case. 

Author information should start 0.3~inches below the bottom rule
surrounding the title. The authors' names should appear in 10~point 
bold type, electronic mail addresses in 10~point small capitals, and
physical addresses in ordinary 10~point type.

Each author's name should be flush left, whereas the email address
should be flush right on the same line. The author's physical address
should appear flush left on the ensuing line, on a single line if
possible. If successive authors have the same affiliation, then give
their physical address only once.

\subsection{Partitioning the Text} 

You should organize your paper into sections and paragraphs to help 
readers place a structure on the material and understand its contributions. 

% Use \vspace for fine control of spacing above and below headings. 
\vspace{-0.018in}
\subsubsection{Sections and Subsections}
\vspace{-0.015in}

Section headings should be numbered, flush left, and set in 11~pt bold
type with the content words capitalized. Leave 0.25~inches of space 
before the heading and 0.15~inches after the heading. 

Similarly, subsection headings should be numbered, flush left, and set
in 10~pt bold type with the content words capitalized. Leave 0.2~inches 
of space before the heading and 0.13~inches afterward.

Finally, subsubsection headings should be numbered, flush left, and set
in 10~pt small caps with the content words capitalized. Leave 0.18~inches 
of space before the heading and 0.1~inches after the heading. Please 
use no more than three levels of headings.

\subsubsection{Paragraphs and Footnotes}

Within each section or subsection, you should further partition the
paper into paragraphs. Do not indent the first line of a given
paragraph, but insert a blank line between succeeding ones.
 
You can use footnotes\footnote{For the sake of readability, footnotes
should be complete sentences.} to provide readers with additional
information about a topic without interrupting the flow of the paper.
Indicate footnotes with a number in the text where the point is most
relevant. Place the footnote in 9~point type at the bottom of the
column in which it appears. Precede the first footnote in a column 
with a horizontal rule of 0.8~inches.\footnote{Multiple footnotes can
appear in each column, in the same order as they appear in the text,
but spread them across columns and pages if possible.}

\begin{figure}[h]
\vskip 0.2in
\begin{center}
\setlength{\epsfxsize}{3.25in}
\centerline{\epsfbox{role.ps}}
% \vskip 0.1in
\caption{Steps in the computational discovery process at which the
         developer can influence system behavior.}
\label{process-flow}
\end{center}
\vskip -0.2in
\end{figure} 

\subsection{Figures}
 
You may want to include figures in the paper to help readers visualize
your approach and your results. Such artwork should be centered,
legible, and separated from the text. Lines should be dark and at
least 0.5~points thick for purposes of reproduction, and text should
not appear on a gray background.

% Use \newpage to insert a page break between paragraphs. 

Label all distinct components of each figure. If the figure takes
the form of a graph, then give a name for each axis and include a 
legend that briefly describes each curve. However, do {\it not\/} 
include a title above the figure, as the caption already serves
this function. 

Number figures sequentially, placing the figure number and caption 
{\it after\/} the graphics, with at least 0.1~inches of space before the
caption and 0.1~inches after it, as in Figure~\ref{process-flow}. 
The figure caption should be set in 9~point type and centered unless
it runs two or more lines, in which case it should be flush left. 
You may float figures to the top or bottom of a column, and you may
set wide figures across both columns, but always place two-column
figures at the top or bottom of the page.

\comment{
% Sample commands to format two figures side by side across the page. 
\begin{figure*}[t]
\hbox{\hskip -0.55in
\setlength{\epsfxsize}{3.25in}
\epsfbox{irrel.ps}
\hskip 0.2in
% \hfill
\setlength{\epsfxsize}{3.25in}
\epsfbox{rel.ps}
}
\vskip 0.1in
\caption{Theoretical and experimental learning curves for naive Bayes
         when (a) the domain involves a `2 of 2' target concept and
         varying numbers of irrelevant attributes, and (b) for a domain
         with one irrelevant attribute and a conjunctive target concept
         with varying numbers of relevant features.}
\label{bayes-curves}
\end{figure*}
}
 
\subsection{Tables} 
 
You may also want to include tables that summarize material. Like 
figures, these should be centered, legible, and numbered consecutively. 
However, place the title {\it above\/} the table with at least 
0.1~inches of space before the title and the same after it, as in 
Table~\ref{sample-table}. The table title should be set in 9~point 
type and centered unless it runs two or more lines, in which case it
should be flush left.

% Note use of \abovespace and \belowspace to get reasonable spacing 
% above and below tabular lines. 

\begin{table}[t]
\caption{Classification accuracies for naive Bayes and flexible 
Bayes on various data sets.}
\label{sample-table}
\vskip 0.15in
\begin{center}
\begin{small}
\begin{sc}
\begin{tabular}{lcccr}
\hline
\abovespace\belowspace
Data set & Naive & Flexible & Better? \\
\hline
\abovespace
Breast    & 95.9$\pm$ 0.2& 96.7$\pm$ 0.2& $\surd$ \\
Cleveland & 83.3$\pm$ 0.6& 80.0$\pm$ 0.6& $\times$\\
Credit    & 74.8$\pm$ 0.5& 78.3$\pm$ 0.6&         \\
Glass2    & 61.9$\pm$ 1.4& 83.8$\pm$ 0.7& $\surd$ \\
Horse     & 73.3$\pm$ 0.9& 69.7$\pm$ 1.0& $\times$\\
Meta      & 67.1$\pm$ 0.6& 76.5$\pm$ 0.5& $\surd$ \\
Pima      & 75.1$\pm$ 0.6& 73.9$\pm$ 0.5&         \\
\belowspace
Vehicle   & 44.9$\pm$ 0.6& 61.5$\pm$ 0.4& $\surd$ \\
\hline
\end{tabular}
\end{sc}
\end{small}
\end{center}
\vskip -0.1in
\end{table}

Tables contain textual material that can be typeset, as contrasted 
with figures, which contain graphical material that must be drawn. 
Specify the contents of each row and column in the table's topmost
row. Again, you may float tables to a column's top or bottom, and set
wide tables across both columns, but place two-column tables at the
top or bottom of the page.
 
\subsection{Citations and References} 

Please use APA reference format regardless of your formatter
or word processor. If you rely on the \LaTeX\/ bibliographic 
facility, use {\tt mlapa.sty} and {\tt mlapa.bst} 
at the ICML-2002 web site to obtain this format.

Citations within the text should include the authors' last names and
year. If the authors' names are included in the sentence, place only
the year in parentheses, as in Jones and VanLehn (1994), but otherwise 
place the entire reference in parentheses with the authors and year
separated by a comma (Jones \& VanLehn, 1984). 

List multiple references alphabetically and separate them by semicolons 
(Jones \& VanLehn, 1984; Veloso \& Carbonell, 1993). Use the `et~al.'
construct only for citations with four or more authors or after listing 
all authors to a publication in an earlier reference.

Use an unnumbered first-level section heading for the references, and 
use a hanging indent style, with the first line of the reference flush
against the left margin and subsequent lines indented by 10 points. 
The references at the end of this document give examples for journal
articles, conference publications, book chapters, books, edited volumes, 
technical reports, and dissertations. 

Alphabetize references by the surnames of the first authors, with
single author entries preceding multiple author entries. Order
references for the same authors by year of publication, with the
earliest first.

\section{Electronic Submission}

ICML-2002 will rely heavily on electronic formats for submission and 
review. We assume that nearly all authors will have access to standard
software for word processing, electronic mail, and ftp file transfer. 
Authors who do not have such access should send email with their concerns 
to {\tt icml2002@cse.unsw.edu.au}. 

\subsection{Templates for Papers}

Electronic templates for producing the camera-ready copy are available
for several major word processors, including \LaTeX\/ and Microsoft Word. 
Templates are accessible on the World Wide Web at:
\vskip 0.1in
\begin{small}
\centerline{{\tt http://www.cse.unsw.edu.au/$\mathtt{\sim}$icml2002/format.html}}
\end{small}
% \vskip 0.1in
\noindent
Send questions about these electronic templates to {\tt
icml2002@cse.unsw.edu.au}.

\subsection{Initial Submission of Papers}

Submission to ICML-2002 will be entirely electronic.  To submit a paper,
please go to the submission web site:
\vskip 0.1in
\begin{small}
\centerline{{\tt http://www.cse.unsw.edu.au/$\mathtt{\sim}$icml2002/cyberchair.html}}
\end{small}

The deadline for submissions to ICML-2002 is {\bf Friday, 1 February
2002}. If your submission does not reach us by this date, it will not
be considered for publication.

To ensure our ability to print submissions, authors must provide their
manuscripts in {\bf postscript} or {\bf pdf} format. If you are preparing
your paper in Word, please use the Apple LaserWriter 16/600~PS driver to
ensure its printability in other environments. If you cannot deliver a 
postscript or pdf file electronically due to exceptional conditions, send
email to {\tt icml2002@cse.unsw.edu.au} to discuss alternative
means of delivery.

ICML-2002 allows simultaneous submission to other conferences, provided 
this fact is clearly indicated on the submission form. Accepted papers 
will appear in the conference proceedings only if withdrawn from other
conferences. Simultaneous submissions that are not clearly specified
as such will be rejected.

\subsection{Submitting Revised Papers}

Final versions of papers accepted for publication, and resubmissions 
of conditionally accepted papers, should follow the same format and
naming convention as initial submissions.

The deadline for revised copies of conditionally accepted papers is
{\bf Monday, 15 April 2002}.  To submit a revised paper for final
review, please go to:
\vskip 0.1in
\begin{small}
\centerline{{\tt http://www.cse.unsw.edu.au/$\mathtt{\sim}$icml2002/submit/}}
\end{small}

Camera-ready copy of all accepted papers (conditional and unconditional)
must be received by the publisher by {\bf Friday, 19 April 2002}. Please
see the Author Kit that has been made available from the ICML-2002 homepage:
\vskip 0.1in
\begin{small}
\centerline{{\tt http://www.cse.unsw.edu.au/$\mathtt{\sim}$icml2002/}}
\end{small}
Please note that the instructions include a Permission to Publish form
that must be signed and sent to the publisher.

After submitting your final copy to the publisher, please be sure to
submit a copy to the conference submission site as well:
\vskip 0.1in
\begin{small}
\centerline{{\tt http://www.cse.unsw.edu.au/$\mathtt{\sim}$icml2002/submit/}}
\end{small}

If your revised paper does not reach us by the due  date, it will not
be included in the proceedings.

\section*{Acknowledgements} 
 
Please place your acknowledgements in an unnumbered section at the
end of the paper. Typically, this will include thanks to reviewers
who gave useful comments, to colleagues who contributed to the ideas, 
and to funding agencies and corporate sponsors that provided financial 
support.  This document was modified by Andrea Danyluk from the file
made available by Pat Langley for ICML-2K.

\section*{References}

{\parindent -10pt\leftskip 10pt

Aha, D.~W.~(1990). {\it A study of instance-based algorithms for
supervised learning tasks: Mathematical, empirical, and psychological
evaluations\/}. Doctoral dissertation, Department of Information \&
Computer Science, University of California, Irvine.

Fisher, D. H. (1989). Noise-tolerant conceptual clustering. {\it
Proceedings of the Eleventh International Joint Conference on Artificial
Intelligence\/} (pp.~825--830). San Francisco: Morgan Kaufmann.

Jones, R.~M., \& VanLehn, K. (1994). Acquisition of children's addition
strategies: A model of impasse-free, knowledge-level learning. {\it
Machine Learning\/}, {\it 16\/}, 11--36.

Langley, P. (1995). {\it Elements of machine learning\/}. San
Francisco: Morgan Kaufmann.

Maloof, M.\ A., Langley, P., Binford, T.\ O., \& Sage, S. (1998).
{\it Improving rooftop detection in aerial images through machine
learning\/} (Technical Report 98-1). Institute for the Study of Learning
and Expertise, Palo Alto, CA.

Shrager, J., \& Langley, P. (Eds.). (1990). {\it Computational models
of scientific discovery and theory formation\/}. San Francisco: Morgan 
Kaufmann.

Veloso, M.\ M., \& Carbonell, J.\ G. (1993). Toward scaling up machine
learning: A case study with derivational analogy in {\sc Prodigy}. In 
S.~Minton (Ed.), {\it Machine learning methods for planning}. San 
Francisco: Morgan Kaufmann.

}
% Leave a blank line before the closing brace to ensure the final 
% reference has the proper indentation. 

\end{document} 
